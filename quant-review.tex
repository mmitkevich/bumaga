\documentclass[a4paper,11pt]{article}
 
\usepackage{times}
\usepackage{natbib}
\usepackage{url}
\usepackage{hyperref}
\usepackage{todo}

\DeclareMathSizes{10}{18}{12}{8}   % For size 10 text
\DeclareMathSizes{11}{19}{13}{9}   % For size 11 text
\DeclareMathSizes{12}{20}{14}{10}  % For size 12 text

\begin{document}
\title{My Quant Review}
\author{Mikhail Mitkevich}
\date{2015-3-15}


\maketitle

\begin{abstract}
keywords: PCA, Markov, ANN, Bayess, Elliott Waves
\end{abstract}

\section{Introduction}
\begin{tabular}{|c|c|c|}
\hline a & b & c \\ 
\hline a1 & b1 &  c1\\ 
\hline 
\end{tabular} 

\section{PCA}

Since \cite{avellaneda2010statistical} using Principal Components Analysis (PCA) to build statistical arbitrage  strategies is quite common \Todo{cites}.
Usually these strategy exploits mean reversion of stock prices to some statistical "equilibrium point", which is determined using PCA.

However, \cite{infantino2010} report that PCA model could behave poor when market switches to new "regime", implying quite different equilibrium state. 
Panic selloff period during financial crisis is example of such regimes. 
\cite{infantino2010} used volatility of PCA vectors as metric to identify these regime switches.
When identified, they used opposite strategy \todo{trading away from PCA}.
After such modification their reported sharp-ratio looked quite good \Todo{sharpe numbers}.




\section{Conclusion}

\section{Future Work}

\todos

\bibliographystyle{te}
\bibliography{research}
\end{document}
